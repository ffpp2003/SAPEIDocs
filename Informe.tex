\documentclass[a4paper, 12pt]{report}
\usepackage[a4paper, %margenes de pagina
  left=2.5cm,
  right=2.5cm,
  top=1.5cm,
  bottom=2cm,
  includehead
]{geometry}
\usepackage{fancyhdr}
\usepackage{etoolbox}
\usepackage{titlesec}
\usepackage{titling} % para personalizar el título
\usepackage{tikz}
\usepackage{pgf}
\usetikzlibrary{positioning}
\UseRawInputEncoding
\usepackage{enumitem}

\usepackage{minted}
\setminted{
    linenos,          % Activa la numeración de líneas
    frame=single,      % Opcional: agrega un marco al código
    framesep=2mm,     % Opcional: separación entre el marco y el texto
    fontsize=\scriptsize,   % Opcional: cambia el tamaño de la fuente
    breaklines
}

\pagestyle{fancy}
\pagestyle{fancy}
\lhead{UTN-FRC}
\chead{Informatica II}
\rhead{2R3}
\cfoot{\thepage}
\setlength{\headwidth}{\textwidth} % Hace que el ancho del encabezado coincida con el ancho del texto
\setlength{\headheight}{10pt}  % Ajusta la altura del encabezado
\setlength{\headsep}{20pt}     % Ajusta la separación entre el encabezado y el contenido
\patchcmd{\chapter}{\thispagestyle{plain}}{\thispagestyle{fancy}}{}{}

\DeclareMathSizes{12}{13}{6}{5}


\title{%
  \fontsize{25}{0}\selectfont Universidad Tecnológica Nacional \\
  \fontsize{22}{30}\selectfont Informatica 2 \\
  \fontsize{18}{25}\selectfont SAPEI\\
  \fontsize{16}{20}\selectfont Sistema Alternativo de Pago para el Estacionamiento Institucional\\
}
\author{
Cortesini Luciano - 402719\\
Gil Ignacio - 401891\\
Grasso Gaston - 401892\\
Noccetti Santino - 405947 \\
Palombo Franco - 401910
}

\date{19 / 11 / 2024}

\titleformat{\chapter}[block]
  {\normalfont\huge\bfseries}{}{0pt}{\Huge}
\titlespacing*{\chapter}{0pt}{-30pt}{20pt}

% Redefinir el formato de las secciones para que solo muestren el título
\titleformat{\section}[block]
  {\normalfont\Large\bfseries}{}{0pt}{\Large}
\titlespacing*{\section}{0pt}{3.5ex plus 1ex minus .2ex}{2.3ex plus .2ex}

% Redefinir el formato de las subsecciones para que solo muestren el título
\titleformat{\subsection}[block]
  {\normalfont\large\bfseries}{}{0pt}{\large}
\titlespacing*{\subsection}{0pt}{3.25ex plus 1ex minus .2ex}{1.5ex plus .2ex}


\begin{document}
\maketitle
\section{Problema y Solucion Planteada}
    El problema que observamos esta en el transito generado por los vehiculos, siendo este generado por la fila de
    autos que hay en el ingreso al estacionamiento de la facultad, lo que obstruye la via publicam reduciendo el flujo
    de los vehiculos y produciendo embotellamientos. La razon por la cual se genera la fila de autosm es por la
    lentitud del sistema de ingreso al estacionamiento. Este sistema requiere de un operario que ingrese la patente
    del vehiculo de manera manual, el cual procede a generar un ticket y automaticamente se le descuenta el monto
    correspondiente al usuario.


    Como grupo se nos ocurrio la idea de acelerar una parte del proceso y es a la hora de pagar, en la cual
    implementando un sistemas Contactless con tarjetas RFID unicas para cada persona con la informacion
    correspondiente. De este modo, no es necesario que el operario tenga que ingresar la patente del vehiculo
    manualmente para poder generar el ticket, ya que se utilizara tanto para el ingreso y egreso la misma tarjeta. De
    esta forma, se puede mantener un conteo de vehiculos en el estacionamiento y ademas evitar la cantidad de papel
    que se utiliza de forma deficiente en un dia.

\section{Planificacion}
    Para poder llevar a cabo esta idea, empezamos por la organizacion de la misma y las necesidades o posibles
    problemas del proyecto. Para resolver lo primero, decidimos usar Github y su sistema ramas creando asi los
    repositorios necesarios para distribuir de forma efectiva las partes de trabajo y asi el proyecto pueda avanzar.
    Creando adentro de los mismos repositorios los problemas y necesidades que surgian, asignandolos asi entre nosotros.

\section{Partes del Proyecto}
    Para resolver la segunda parte se penso por dividir el proyecto en proyecto en 4 partes, teniendo asi:
    \begin{enumerate}
        \item Hardware
        \item Firmware
        \item Core
        \item Client
    \end{enumerate}
    \newpage

\section{Hardware}
    Los componentes usados para el hardware fueron:
    \begin{enumerate}
        \item Arduino Nano
        \item RC522 (Modulo RFID)
        \item transistor NPN BC548
        \item 3 Leds
        \item Buzzer Activo
        \item Protoboard
        \item Resistencias y cables
    \end{enumerate}

\section{Diagrama Funcional}
    \begin{center}
    \begin{tikzpicture}[auto, node distance=2cm, >=stealth, line width=0.8mm]
        % Definición de bloques
        \node [draw, rectangle, minimum width=2.5cm, minimum height=1.5cm] (block1) {Modulo RFID};
        \node [draw, rectangle, minimum width=2.5cm, minimum height=1.5cm, right of=block1, node distance=4cm] (block2) {Arduino NANO};
        \node [draw, rectangle, minimum width=2.5cm, minimum height=1.5cm, right of=block2, node distance=4cm] (block3) {Client};
        \node [draw, rectangle, minimum width=2.5cm, minimum height=1.5cm, right of=block3, node distance=4cm] (block4) {Base de Datos};
        \node [draw, rectangle, minimum width=2.5cm, minimum height=1.5cm, above of=block1, node distance=4cm] (block5) {Usuario};
    
        % Flechas más grandes
        \draw[->, >=stealth, line width=1mm] (block1) -- (block2); % Grosor de la línea ajustado
        \draw[<->] (block2) -- (block3);
        \draw[<->] (block3) -- (block4);
        \draw[->, >=stealth, line width=1mm] (block5) -- (block1);
    \end{tikzpicture}
    \end{center}

\newpage
\section{Firmware}
\begin{enumerate}[left=0pt]
    \item SAPEIFirwmare.ino
    \inputminted{c++}{../SAPEIFirmware/SAPEIFirmware.ino}
    \newpage
    \item gpio.h:
    \inputminted{c++}{../SAPEIFirmware/src/lib/gpio.h}
    \item gpio.cpp:
    \inputminted{c++}{../SAPEIFirmware/src/lib/gpio.cpp}
\end{enumerate}

\newpage
\section{Core}
\begin{enumerate}[left=0pt]
    \item Client.h
    \inputminted{c++}{../SAPEICore/Client.h}
    \item Client.cpp
    \inputminted{c++}{../SAPEICore/Client.cpp}
    \newpage
    \item Vehicle.h
    \inputminted{c++}{../SAPEICore/Vehicle.h}
    \item Vehicle.cpp
    \inputminted{c++}{../SAPEICore/Vehicle.cpp}
    \newpage
    \item Person.h
    \inputminted{c++}{../SAPEICore/Person.h}
    \item Person.cpp
    \inputminted{c++}{../SAPEICore/Person.cpp}
    \newpage
    \item Error.h
    \inputminted{c++}{../SAPEICore/Error.h}
    \item Error.cpp
    \inputminted{c++}{../SAPEICore/Error.cpp}
    \newpage
    \item Database.h
    \inputminted{c++}{../SAPEICore/DataBase.h}
    \item Database.cpp
    \inputminted{c++}{../SAPEICore/DataBase.cpp}
\end{enumerate}

\newpage
\section{Client}
\begin{enumerate}[left=0pt]
    \item main.cpp
    \inputminted{c++}{../SAPEIClient/main.cpp}
    \newpage
    \item addcarddialog.h
    \inputminted{c++}{../SAPEIClient/addcarddialog.h}
    \item addcarddialog.cpp
    \inputminted{c++}{../SAPEIClient/addcarddialog.cpp}
    \newpage
    \item addvehicledialog.h
    \inputminted{c++}{../SAPEIClient/addvehicledialog.h}
    \item addvehicledialog.cpp
    \inputminted{c++}{../SAPEIClient/addvehicledialog.cpp}
    \newpage
    \item balancehandler.h
    \inputminted{c++}{../SAPEIClient/balancehandler.h}
    \item balancehandler.cpp
    \inputminted{c++}{../SAPEIClient/balancehandler.cpp}
    \newpage
    \item clientlistdialog.h
    \inputminted{c++}{../SAPEIClient/clientlistdialog.h}
    \item clientlistdialog.cpp
    \inputminted{c++}{../SAPEIClient/clientlistdialog.cpp}
    \newpage
    \item editclientdialog.h
    \inputminted{c++}{../SAPEIClient/editclientdialog.h}
    \item editclientdialog.cpp
    \inputminted{c++}{../SAPEIClient/editclientdialog.cpp}
    \newpage
    \item editvehicledialog.h
    \inputminted{c++}{../SAPEIClient/editvehicledialog.h}
    \item editvehicledialog.cpp
    \inputminted{c++}{../SAPEIClient/editvehicledialog.cpp}
    \newpage
    \item mainwindow.h
    \inputminted{c++}{../SAPEIClient/mainwindow.h}
    \item mainwindow.cpp
    \inputminted{c++}{../SAPEIClient/mainwindow.cpp}
    \newpage
    \item serialhandler.h
    \inputminted{c++}{../SAPEIClient/serialhandler.h}
    \item serialhandler.cpp
    \inputminted{c++}{../SAPEIClient/serialhandler.cpp}
    \newpage
    \item vehiclelistdialog.h
    \inputminted{c++}{../SAPEIClient/vehiclelistdialog.h}
    \item vehiclelistdialog.cpp
    \inputminted{c++}{../SAPEIClient/vehiclelistdialog.cpp}
\end{enumerate}
\end{document}
